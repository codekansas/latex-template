\section{Method}

This section presents our proposed methodology in detail. We describe the overall approach, key innovations, and implementation details that enable our system to address the challenges identified in the related work.

\subsection{Overview}

Our method consists of three main components: (1) a novel preprocessing pipeline, (2) an innovative neural architecture, and (3) a robust training procedure. % The overall workflow is illustrated in Figure \ref{fig:overview}.

% \begin{figure}[h]
% \centering
% \includegraphics[width=0.8\textwidth]{figures/method_overview.png}
% \caption{Overview of our proposed method showing the three main components and their interactions.}
% \label{fig:overview}
% \end{figure}

% Note: Figure placeholder - add method_overview.png to figures/ directory when available

\subsection{Preprocessing Pipeline}

The preprocessing pipeline transforms raw input data into a format suitable for our neural architecture. This involves several key steps:

\subsubsection{Data Normalization}

We apply z-score normalization to ensure all features have zero mean and unit variance:

\begin{equation}
x_{norm} = \frac{x - \mu}{\sigma}
\end{equation}

where $\mu$ and $\sigma$ are the mean and standard deviation of the training data, respectively.

\subsubsection{Feature Engineering}

We extract several types of features:

\begin{itemize}
    \item \textbf{Statistical features}: Mean, variance, skewness, and kurtosis
    \item \textbf{Frequency domain features}: FFT coefficients and power spectral density
    \item \textbf{Temporal features}: First and second derivatives, moving averages
    \item \textbf{Domain-specific features}: Custom features based on domain knowledge
\end{itemize}

\subsubsection{Data Augmentation}

To improve generalization, we apply several augmentation techniques:

\begin{enumerate}
    \item \textbf{Noise injection}: Add Gaussian noise with $\sigma = 0.01$
    \item \textbf{Time warping}: Apply random time stretching/compression
    \item \textbf{Magnitude scaling}: Randomly scale feature magnitudes
    \item \textbf{Label smoothing}: Apply label smoothing with $\alpha = 0.1$
\end{enumerate}

\subsection{Neural Architecture}

Our neural architecture is designed to efficiently process the preprocessed data while maintaining interpretability. The architecture consists of several key components:

\subsubsection{Input Layer}

The input layer accepts preprocessed features of dimension $d$ and applies an initial transformation:

\begin{equation}
h_0 = \text{ReLU}(W_0 x + b_0)
\end{equation}

where $W_0 \in \mathbb{R}^{d \times h}$ and $b_0 \in \mathbb{R}^h$ are learnable parameters.

\subsubsection{Attention Mechanism}

We employ a multi-head attention mechanism to capture long-range dependencies:

\begin{equation}
\text{Attention}(Q, K, V) = \text{softmax}\left(\frac{QK^T}{\sqrt{d_k}}\right)V
\end{equation}

where $Q$, $K$, and $V$ are query, key, and value matrices, respectively.

\subsubsection{Residual Connections}

Residual connections help with gradient flow and enable deeper networks:

\begin{equation}
h_{l+1} = h_l + \text{LayerNorm}(\text{FFN}(h_l))
\end{equation}

where FFN is a feed-forward network and LayerNorm is layer normalization.

\subsubsection{Output Layer}

The output layer produces the final predictions:

\begin{equation}
\hat{y} = \text{softmax}(W_{out} h_L + b_{out})
\end{equation}

where $h_L$ is the output of the final hidden layer.

\subsection{Training Procedure}

Our training procedure incorporates several techniques to improve performance and robustness:

\subsubsection{Loss Function}

We use a combination of cross-entropy loss and regularization terms:

\begin{equation}
\mathcal{L} = \mathcal{L}_{CE} + \lambda_1 \mathcal{L}_{L2} + \lambda_2 \mathcal{L}_{attention}
\end{equation}

where:
\begin{itemize}
    \item $\mathcal{L}_{CE}$ is the standard cross-entropy loss
    \item $\mathcal{L}_{L2}$ is L2 regularization on model parameters
    \item $\mathcal{L}_{attention}$ is a regularization term on attention weights
    \item $\lambda_1$ and $\lambda_2$ are hyperparameters
\end{itemize}

\subsubsection{Optimization}

We use the Adam optimizer with the following hyperparameters:

\begin{itemize}
    \item Learning rate: $10^{-3}$ with cosine annealing
    \item Batch size: 32
    \item Weight decay: $10^{-4}$
    \item Gradient clipping: Maximum norm of 1.0
\end{itemize}

\subsubsection{Training Schedule}

The training process follows a specific schedule:

\begin{enumerate}
    \item \textbf{Warmup phase} (10\% of total epochs): Linear learning rate warmup
    \item \textbf{Main training phase} (80\% of total epochs): Full learning rate
    \item \textbf{Fine-tuning phase} (10\% of total epochs): Reduced learning rate
\end{enumerate}

\subsection{Implementation Details}

Our implementation is built using PyTorch and follows best practices for reproducibility:

\begin{itemize}
    \item Random seeds are fixed for reproducibility
    \item All experiments are run on identical hardware configurations
    \item Model checkpoints are saved at regular intervals
    \item Comprehensive logging of training metrics and hyperparameters
\end{itemize}

\subsection{Computational Complexity}

The computational complexity of our method is:

\begin{itemize}
    \item \textbf{Time complexity}: $O(n \cdot d^2)$ where $n$ is sequence length and $d$ is feature dimension
    \item \textbf{Space complexity}: $O(n \cdot d)$ for storing intermediate representations
    \item \textbf{Training time}: Approximately 2 hours on a single GPU
    \item \textbf{Inference time}: 10ms per sample on average
\end{itemize}

This methodology provides a solid foundation for the experimental evaluation presented in the next section.
